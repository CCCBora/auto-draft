\section{introduction}
Deep learning has shown remarkable success in various fields, including image and text recognition, natural language processing, and computer vision. However, the challenge of overfitting persists, especially in real-world applications where data may be scarce or noisy \cite{2010.05244}. Adversarial training has emerged as a promising technique to improve the robustness and generalization ability of neural networks, making them more resistant to adversarial examples \cite{2108.08976}. In this paper, we propose a novel approach to training adversarial generative neural networks using an adaptive dropout rate, which aims to address the overfitting issue and improve the performance of deep neural networks (DNNs) in various applications.

Dropout has been a widely-used regularization technique for training robust deep networks, as it effectively prevents overfitting by avoiding the co-adaptation of feature detectors \cite{1911.12675}. Various dropout techniques have been proposed, such as binary dropout, adaptive dropout, and DropConnect, each with its own set of advantages and drawbacks \cite{1805.10896}. However, most existing dropout methods are input-independent and do not consider the input data while setting the dropout rate for each neuron. This limitation makes it difficult to sparsify networks without sacrificing accuracy, as each neuron must be generic across inputs \cite{1805.10896, 2212.14149}.

In our proposed solution, we extend the traditional dropout methods by incorporating an adaptive dropout rate that is sensitive to the input data. This approach allows each neuron to evolve either to be generic or specific for certain inputs, or dropped altogether, which in turn enables the resulting network to tolerate a higher degree of sparsity without losing its expressive power \cite{2004.13342}. We build upon the existing work on advanced dropout \cite{2010.05244}, variational dropout \cite{1805.10896}, and adaptive variational dropout \cite{1805.08355}, and introduce a novel adaptive dropout rate that is specifically designed for training adversarial generative neural networks.

Our work differs from previous studies in several ways. First, we focus on adversarial generative neural networks, which have shown great potential in generating realistic images and other forms of data \cite{2303.15533}. Second, we propose an adaptive dropout rate that is sensitive to the input data, allowing for better sparsification and improved performance compared to input-independent dropout methods \cite{1805.10896, 2212.14149}. Finally, we demonstrate the effectiveness of our approach on a variety of applications, including image generation, text classification, and regression, showing that our method outperforms existing dropout techniques in terms of accuracy and robustness \cite{2010.05244, 2004.13342}.

In conclusion, our research contributes to the ongoing efforts to improve the performance and robustness of deep learning models, particularly adversarial generative neural networks. By introducing an adaptive dropout rate that is sensitive to the input data, we aim to address the overfitting issue and enhance the generalization ability of these networks. Our work builds upon and extends the existing literature on dropout techniques and adversarial training, offering a novel and promising solution for training more robust and accurate deep learning models in various applications.