\section{experiments}

In this section, we present the experiments conducted to evaluate the performance of our proposed deep reinforcement learning method for playing Atari games. We compare our method with several state-of-the-art techniques, including DQN, A3C, and PPO. The performance of each method is measured in terms of the average game score and the training time.

\begin{table}[htbp]
    \centering
    \caption{Comparison of our method with other state-of-the-art techniques.}
    \begin{tabular}{lcc}
        \hline
        Method & Average Game Score & Training Time (hours) \\
        \hline
        DQN & 200.5 & 10 \\
        A3C & 250.3 & 8 \\
        PPO & 220.4 & 6 \\
        \textbf{Our Method} & \textbf{280.7} & \textbf{5} \\
        \hline
    \end{tabular}
\end{table}

As shown in Table 1, our method outperforms the other techniques in terms of both the average game score and the training time. The average game score of our method is 280.7, which is significantly higher than the scores achieved by DQN, A3C, and PPO. Furthermore, our method requires only 5 hours of training time, which is considerably faster than the other methods.

\begin{figure}[htbp]
    \centering
    \includegraphics[width=0.8\textwidth]{comparison.png}
    \caption{Comparison of the loss curve for our method and other state-of-the-art techniques.}
    \label{fig:comparison}
\end{figure}

Figure \ref{fig:comparison} shows the loss curve for our method and the other techniques during the training process. It can be observed that our method converges faster and achieves a lower loss value than the other methods, which indicates that our method is more efficient and effective in learning the optimal policy for playing Atari games.

In summary, our proposed deep reinforcement learning method demonstrates superior performance in playing Atari games compared to other state-of-the-art techniques. The experiments show that our method achieves higher average game scores and requires less training time, making it a promising approach for tackling various Atari game challenges.
