\section{introduction}
Deep reinforcement learning (DRL) has shown remarkable success in various domains, including finance, medicine, healthcare, video games, robotics, and computer vision \cite{2108.11510}. One of the most notable applications of DRL is in playing Atari games, where agents learn to play directly from raw pixels \cite{1708.05866}. The motivation for this research is to advance the field of artificial intelligence by developing a DRL agent capable of playing Atari games with improved performance and efficiency. This area of research is of significant importance and relevance to the AI community, as it serves as a stepping stone towards constructing intelligent autonomous systems that offer a better understanding of the visual world \cite{1709.05067}.

The primary problem addressed in this paper is the development of a DRL agent that can efficiently and effectively learn to play Atari games. Our proposed solution involves employing state-of-the-art DRL algorithms and techniques, focusing on both representation learning and behavioral learning aspects. The specific research objectives include investigating the performance of various DRL algorithms, exploring strategies for improving sample efficiency, and evaluating the agent's performance in different Atari game environments \cite{2212.00253}.

Key related work in this field includes the development of deep Q-networks (DQNs) \cite{1708.05866}, trust region policy optimization (TRPO) \cite{1708.05866}, and asynchronous advantage actor-critic (A3C) algorithms \cite{1709.05067}. These works have demonstrated the potential of DRL in playing Atari games and have laid the groundwork for further research in this area. However, there is still room for improvement in terms of sample efficiency, generalization, and scalability.

The main differences between our work and the existing literature are the incorporation of novel techniques and strategies to address the challenges faced by DRL agents in playing Atari games. Our approach aims to improve sample efficiency, generalization, and scalability by leveraging recent advancements in DRL, such as environment modeling, experience transfer, and distributed modifications \cite{2212.00253}. Furthermore, we will evaluate our proposed solution on a diverse set of Atari game environments, providing a comprehensive analysis of the agent's performance and robustness.

In conclusion, this paper aims to contribute to the field of AI by developing a DRL agent capable of playing Atari games with improved performance and efficiency. By building upon existing research and incorporating novel techniques, our work has the potential to advance the understanding of DRL and its applications in various domains, ultimately paving the way for the development of more intelligent and autonomous systems in the future. 