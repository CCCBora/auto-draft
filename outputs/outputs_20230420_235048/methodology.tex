\section{methodology}
\subsection{Deep Convolutional Neural Network}
Our proposed model employs a deep convolutional neural network (CNN) to process the raw pixel inputs from the Atari game environment. The CNN is composed of multiple convolutional layers with ReLU activation functions, followed by fully connected layers. The architecture is designed to efficiently extract high-level features from the raw pixel inputs, which are then used as input for the Q-learning algorithm. The CNN is defined as follows:
\[f_{\theta}(s) = \phi(W^{(L)}\sigma(W^{(L-1)}\dots\sigma(W^{(1)}s + b^{(1)})\dots) + b^{(L)})\]
where $f_{\theta}(s)$ is the output of the CNN, $\theta = \{W^{(i)}, b^{(i)}\}_{i=1}^L$ are the weights and biases of the network, $L$ is the number of layers, $\sigma$ is the ReLU activation function, and $\phi$ is the final activation function.

\subsection{Q-Learning with Experience Replay and Target Networks}
To estimate the action-value function, we employ a Q-learning algorithm combined with experience replay and target networks. Experience replay stores the agent's past experiences in a replay buffer $\mathcal{D}$, which is then used to sample mini-batches for training. This approach helps to break the correlation between consecutive samples and stabilize the training process. The target network is a separate network with parameters $\theta^{-}$ that are periodically updated from the main network's parameters $\theta$. This technique further stabilizes the training by providing a fixed target for the Q-learning updates. The Q-learning update rule is given by:
\[\theta \leftarrow \theta + \alpha (r + \gamma \max_{a'} Q(s', a'; \theta^{-}) - Q(s, a; \theta))\nabla_{\theta} Q(s, a; \theta)\]
where $\alpha$ is the learning rate, and the other variables are as previously defined.

\subsection{Training and Evaluation}
We train our proposed model using the following procedure: The agent interacts with the Atari game environment, and the raw pixel inputs are processed by the CNN to obtain high-level features. The agent then selects an action based on an $\epsilon$-greedy exploration strategy, where $\epsilon$ is the exploration rate. The agent receives a reward and the next state, and the experience is stored in the replay buffer. Periodically, the agent samples a mini-batch from the replay buffer and updates the network parameters using the Q-learning update rule. The target network parameters are updated every $C$ steps.

To evaluate our model, we follow the protocol established in previous works \cite{1708.05866}. We test the agent's performance on a diverse set of Atari game environments and compare the results with state-of-the-art DRL algorithms and human players. The evaluation metrics include average episode reward, human-normalized score, and training time. Additionally, we analyze the agent's ability to generalize across different games and its sample efficiency compared to existing methods. This comprehensive evaluation will provide insights into the robustness and effectiveness of our proposed approach in playing Atari games using deep reinforcement learning.